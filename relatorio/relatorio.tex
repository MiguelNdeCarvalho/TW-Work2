\documentclass[11pt]{article}   % tipo de documento e tamanho das letras

% os seguintes pacotes estendem a funcionalidade básica:
\usepackage[a4paper, total={16cm, 24cm}]{geometry} % tamanho da pagina e do texto
\usepackage[portuguese]{babel}  % traduz para portugues
\usepackage[utf8]{inputenc}
\usepackage{graphicx}           % graficos
\usepackage{amsmath}            % matematica
\usepackage{tikz}               % diagramas
    \usetikzlibrary{shadows}
\usepackage{booktabs}           % tabelas com  melhor aspecto
\usepackage[colorlinks=true]{hyperref}           % links para partes do documento ou para a web
\usepackage{listings}           % incluir codigo
    \renewcommand\lstlistingname{Listagem}  % Listing em portugues
    \lstset{numbers=left, numberstyle=\tiny, numbersep=5pt, basicstyle=\footnotesize\ttfamily, frame=tb,rulesepcolor=\color{gray}, breaklines=true}
\usepackage{blindtext}

% -------------------------------------------------------------------------------------------
\title
{
    \includegraphics[width=0.3\textwidth]{images/logo_universidade.png}
    \\[0.1cm]
    \textbf{Aplicação Web - 2ª Parte} \\
    Tecnologias Web
}

\author
{
    \textbf{Professores:} José Saias, Pedro Salgueiro \\
    \textbf{Realizado por:} Miguel de Carvalho (43108), João Pereira (42864) 
}
\date{\today}

% -------------------------------------------------------------------------------------------
%                                Body                                                       %
% -------------------------------------------------------------------------------------------

\begin{document}
\maketitle

% -------------------------------------------------------------------------------------------
\section{Introdução} 

\hspace{0,5cm}Na 1ª parte do trabalho foi solicitado a realização de uma base sólida para criar uma 
\textbf{aplicação web de venda de produtos online} usando \textbf{HTML} e \textbf{CSS}, ou seja,
realizar o \textbf{Frontend}. \par

Nesta 2ª parte do trabalho foi solicitado a Implementação do \textbf{Backend} da \textbf{aplicação web de 
venda de produtos online}, usando o \textbf{Frontend} desenvolvido na 1ª parte do trabalho, por forma a 
obter uma aplicação web utilizável.
% -------------------------------------------------------------------------------------------
\section{Implementação}

\hspace{0,5cm}Inicialmente pensámos como deveríamos proceder para realizar o trabalho e decidimos que
devíamos começar por implementar a \textbf{autenticação} (utilizadores).

\subsection{Implementação dos Utilizadores}

Os \textbf{Utilizadores} são constituídos por 6 atributos:
\begin{itemize}
    \item \verb|firstName| - primeiro nome do utilizador;
    \item \verb|lastName| - último nome do utilizador;
    \item \verb|mail| - mail do utilizador;
    \item \verb|username| - username do utilizador;
    \item \verb|password| - password do utilizador;
    \item \verb|role| - papel do utilizador, pode ser \textbf{User ou Admin}.
\end{itemize}

Para implementarmos a \textbf{autenticação}, tivemos que implementar 1 \textbf{controlador}:
\begin{itemize}
    \item \verb|NewUserController| - adicionar utilizadores;
\end{itemize}


\subsection{Implementação dos Produtos}

De seguida, decidimos fazer a implementação dos \textbf{Produtos}. 

São constituídos por 4 atributos:
\begin{itemize}
    \item \verb|name| - título do produto;
    \item \verb|imgPath| - nome da imagem com a extensão do ficheiro, guardada na pasta 
    \verb|/imgs/products|;
    \item \verb|desc| - descrição do produto;
    \item \verb|price| - preço do produto.
\end{itemize}

Para implementarmos os \textbf{produtos} tivemos que implementar  \textbf{controladores}: 
\begin{itemize}
    \item \verb|NewProductController| - adiciona produtos;
    \item \verb|DeleteProductController| - remove produtos;
    \item \verb|ShowProductController| - mostra todos os produtos.
\end{itemize}


\subsection{Implementação das Encomendas}

Depois começamos por fazer a implementação das \textbf{encomendas} e realizar a conexão existente
entre as \textbf{encomendas} e os \textbf{utilizadores} e os \textbf{produtos}.

As \textbf{encomendas} são constituídos por 4 atributos:
\begin{itemize}
    \item \verb|user| - utilizador a quem pertence a encomenda;
    \item \verb|product| - produto da encomenda;
    \item \verb|quantity| - quantidade do produto;
    \item \verb|price| - preço da encomenda;
    \item \verb|date| - data em que a encomenda foi realizada.
\end{itemize}

Para controlar o nosso repositório de \textbf{encomendas} tivemos que implementar 2 \textbf{controladores}:
\begin{itemize}
    \item \verb|ListOrdersController| - mostra todas as encomendas realizadas;
    \item \verb|NerOrdersController| - cria uma encomenda.
\end{itemize}

\subsection{Passos finais}

Por fim, \textbf{implementámos} a página de administração e configurámos as \textbf{permissões} das páginas e 
confirmamos se os utilizadores conseguiam aceder às páginas que só \textbf{utilizadores registados
sem privilégios de administrador} tinham acesso e se não tinham acesso às páginas dos \textbf{administradores} 
e se todos os links posteriormente feitos na 1ª parte do trabalho estavam a redirecionar para onde deveriam.

% -------------------------------------------------------------------------------------------
\section{Dificuldades}

As maiores dificuldades sentidas durante a realização deste trabalho, foram maioritariamente
relacionadas com o funcionamento do \textbf{Springboot} e como é realizada a manipulação dos
dados e como isso se reflete no funciona da \textbf{Base de Dados}.

% -------------------------------------------------------------------------------------------
\section{Conclusão} % Conclusão

\hspace{0,5cm}Assim com a realização desta 2ª parte juntamente com a 1ª parte, conseguimos criar 
uma \textbf{Aplicação Web} totalmente funcional que representa uma \textbf{aplicação web 
de venda de produtos online}.

Em suma, com a realização deste trabalho ficámos muito mais esclarecidos de como é
realizada o \textbf{Backend} de uma \textbf{Aplicação Web} de raiz e quais os requisitos para tal.\par
Saliento também que neste trabalho aplicámos todo o conhecimento que adquirimos durante as aulas e
ajudou a compreender alguns pontos que não tínhamos entendido tão bem.
% -------------------------------------------------------------------------------------------
\end{document}