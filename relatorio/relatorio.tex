\documentclass[11pt]{article}   % tipo de documento e tamanho das letras

% os seguintes pacotes estendem a funcionalidade básica:
\usepackage[a4paper, total={16cm, 24cm}]{geometry} % tamanho da pagina e do texto
\usepackage[portuguese]{babel}  % traduz para portugues
\usepackage[utf8]{inputenc}
\usepackage{graphicx}           % graficos
\usepackage{amsmath}            % matematica
\usepackage{tikz}               % diagramas
    \usetikzlibrary{shadows}
\usepackage{booktabs}           % tabelas com  melhor aspecto
\usepackage[colorlinks=true]{hyperref}           % links para partes do documento ou para a web
\usepackage{listings}           % incluir codigo
    \renewcommand\lstlistingname{Listagem}  % Listing em portugues
    \lstset{numbers=left, numberstyle=\tiny, numbersep=5pt, basicstyle=\footnotesize\ttfamily, frame=tb,rulesepcolor=\color{gray}, breaklines=true}
\usepackage{blindtext}

% -------------------------------------------------------------------------------------------
\title
{
    \includegraphics[width=0.3\textwidth]{images/logo_universidade.png}
    \\[0.1cm]
    \textbf{Aplicação Web} \\
    Tecnologias Web
}

\author
{
    \textbf{Professores:} José Saias, Pedro Salgueiro \\
    \textbf{Realizado por:} Miguel de Carvalho (43108), João Pereira (42864) 
}
\date{\today}

% -------------------------------------------------------------------------------------------
%                                Body                                                       %
% -------------------------------------------------------------------------------------------

\begin{document}
\maketitle

% -------------------------------------------------------------------------------------------
\section{Introdução} 

\hspace{0,5cm}Neste trabalho foi solicitado a realização de uma base sólida para criar uma 
\textbf{aplicação web de venda de produtos online} usando \textbf{HTML} e \textbf{CSS}. \par
HTML \textbf{(Hypertext Markup Language)} é uma linguagem de \textbf{Markup} usada para a criação de páginas \textbf{Web}.
CSS \textbf{(Cascading Style Sheets)} é a linguagem usada para definir o estilo de um documento \textbf{HTML}. \par
A \textbf{base sólida} é constituída por \textbf{três páginas web}:
\begin{itemize}
    \item \textbf{Lista de produtos} 
    \item \textbf{Informação de produtos} 
    \item \textbf{Pesquisa} 
\end{itemize}

Na página \textbf{lista de produtos}, deverão ser listados os produtos disponíveis para venda ou os produtos
resultantes de uma pesquisa.

Na página \textbf{informação de produtos} deverá ser apresentada a informação detalhada de um produto;

Na página \textbf{pesquisa} deverá ser apresentada um menu para efetuar uma pesquisa avançada.

A \textbf{aplicação web} deverá também estar preparada para poder ser usada por pessoas com necessidades especiais.
Além disso deverá ser responsiva, devendo adaptar-se ao dispositivo que o utilizador usa para aceder.
% -------------------------------------------------------------------------------------------
\section{Implementação}

\hspace{0,5cm}Inicialmente pensámos como deveríamos proceder para realizar o trabalho e decidimos que
devíamos começar por fazer um esquema, uma ideia, de como iríamos querer que a nossa \textbf{Aplicação Web} fosse.

\subsection{Criação das Páginas}

Em seguida começamos por fazer a criação da estrutura das páginas que eram essenciais para a criação da \textbf{Aplicação Web}.

\subsubsection{Página Lista de Produtos}

A página \textbf{Lista de Produtos}, que é a nossa página inicial, lista os produtos disponíveis para venda ou os produtos
resultantes de uma pesquisa. Corresponde ao ficheiro \verb|index.html|.
É constituída por 4 secções:
\begin{itemize}
    \item Cabeçalho - possui o logótipo da \textbf{aplicação web}, a barra de pesquisa, um botão de pesquisa e um botão que redireciona para a pesquisa avançada;
    \item Categorias - lista as categorias dos produtos disponíveis;
    \item Produtos - lista os produtos disponíveis na loja, ou os produtos resultantes de uma pesquisa;
    \item About - lista informação genérica sobre a página.
\end{itemize}

\subsubsection{Página Informação de Produtos}

A página \textbf{Informação de Produtos}, apresenta a informação detalhada de um produto.
Corresponde ao ficheiro \verb|product.html|
É constituída por 4 secções:
\begin{itemize}
    \item Cabeçalho - possui o logótipo da \textbf{aplicação web}, a barra de pesquisa, um botão de pesquisa e um botão que redireciona para a pesquisa avançada;
    \item Categorias - lista as categorias dos produtos disponíveis;
    \item Detalhes do produto - lista a informação detalhada de um produto, tais como:
        \begin{itemize}
            \item preço do produto;
            \item os custos de envio;
            \item a data esperada de entrega; 
            \item tipos de pagamento disponíveis.
        \end{itemize}
    \item About - lista informação genérica sobre a página.
\end{itemize}

\subsubsection{Página Pesquisa}

A página \textbf{Pesquisa}, apresenta um menu para efetuar uma pesquisa avançada.
Corresponde ao ficheiro \verb|advanced_search.html|
É constituída por 4 secções:
\begin{itemize}
    \item Cabeçalho - possui o logótipo da \textbf{aplicação web}, a barra de pesquisa, um botão de pesquisa e um botão que redireciona para a pesquisa avançada;
    \item Pesquisa Avançada - permite aos utilizadores realizar pesquisas avançadas de produtos, envolve estes campos:
        \begin{itemize}
            \item descrição do produto;
            \item categoria do produto;
            \item intervalo de preços;
            \item localização do produto;
            \item data em que o produto foi listado na loja pela 1ª vez;
            \item data em que o produto deixa de estar disponível para venda.
        \end{itemize}
    \item About - lista informação genérica sobre a página.
\end{itemize}

\subsection{Adicionar estilos ás páginas}
De seguida começamos por adicionar estilos às páginas anteriormente criadas, através do \textbf{CSS},
sem ter em vista o comportamento da página em dispositivos com diferentes larguras.\par


Estes estilos vão fazer com que as páginas tenham um melhor aspeto visual e sejam apresentadas
da forma que nós queremos.

\subsection{Adicionar responsividade ás páginas}

Por último, vamos adicionar responsividade ás páginas criadas, através do \textbf{CSS}, 
para que possam suportar dispositivos diferentes. \par
Para adicionar responsividade precisamos de adicionar condições para quando o dispositivo
tem uma certa \textbf{largura}. Ou seja, quando um dispositvo tiver uma certa \textbf{largura}, irá
usar aqueles estilos e quando outro dispositivo tiver outra \textbf{largura} irá ter outros estilos.

\subsubsection{Seccção Produtos}
Condições que foram adicionadas ao \textbf{CSS} para adicionar responsividade à \textbf{secção produtos}:
\begin{itemize}
    \item Se a largura do dispositivo do utilizador tiver mais de \textbf{480px} e menos de \textbf{600px}: 
    \begin{itemize}
        \item a imagem deverá ter uma largura de \textbf{124px};
    \end{itemize}
    \item Se a largura do dispositivo do utilizador tiver entre \textbf{690px} e \textbf{690px}:
    \begin{itemize}
        \item a imagem deverá ter uma largura de \textbf{180px};
    \end{itemize}
    \item Se a largura do dispositivo do utilizador tiver mais de \textbf{690px}:
    \begin{itemize}
        \item a imagem deverá ter uma largura de \textbf{227px};
    \end{itemize}
\end{itemize}

\subsubsection{Seccção Categorias}
Condições que foram adicionadas ao \textbf{CSS} para adicionar responsividade à \textbf{secção categorias}:
\begin{itemize}
    \item Se a largura do dispositivo do utilizador tiver menos de \textbf{750px}:
    \begin{itemize}
        \item A \textbf{secção categorias} deverá ter a largura da página.
    \end{itemize}
    \item Se a largura do dispositivo do utilizador tiver mais de \textbf{750px}:
    \begin{itemize}
        \item A \textbf{secção categorias} deverá ter uma largura de \textbf{205px}.
    \end{itemize} 
\end{itemize}

\subsubsection{Secção Informação de Produtos}
\begin{itemize}
    \item Se a largura do dispositivo do utilizador tiver menos de \textbf{600px}:
    \begin{itemize}
        \item A imagem do produto deverá ter uma largura de \textbf{550px};
        \item A descrição do produto deve ficar por baixo da imagem.
    \end{itemize}
    \item Se a largura do dispositivo do utilizador tiver mais de \textbf{600px}:
    \begin{itemize}
        \item A imagem do produto deverá ter uma largura de \textbf{300px};
        \item A descrição do produto deve contornar a imagem pela direita.
    \end{itemize}
\end{itemize}

\subsubsection{Resto das condições}
\begin{itemize}
    \item Se a largura do dispositivo do utilizador tiver menos de \textbf{750px}:
    \begin{itemize}
        \item A \textbf{secção produtos} deverá ocupar toda a largura da página;
        \item A \textbf{secção categorias} deverá estar depois da \textbf{secção produtos}, ocupando toda a largura da página.
    \end{itemize}
    \item Se a largura do dispositivo do utilizador tiver mais de \textbf{750px}:
    \begin{itemize}
        \item A \textbf{secção categorias} deverá estar à esquerda da \textbf{secção produtos};
        \item A \textbf{secção categorias} deverá ter uma largurda de \textbf{205px};
        \item A \textbf{secção produtos} deve ocupar o resto da largura;
    \end{itemize}
\end{itemize}
% -------------------------------------------------------------------------------------------
\section{Dificuldades}

As maiores dificuldades sentidas durante a realização deste trabalho, foram maioritariamente
relacionadas com os \textbf{estilos das páginas (CSS)}, mais propriamente com a \textbf{responsividade} das
páginas.

% -------------------------------------------------------------------------------------------
\section{Conclusão} % Conclusão

\hspace{0,5cm}Em suma, com a realização deste trabalho "Aplicação Web" ficámos muito mais esclarecidos de como é
realizada uma \text{Aplicação Web} de raiz e quais os requisitos para tal. \par
Saliento que neste trabalho aplicámos todo o conhecimento que adquirimos durante as aulas e que ajudou
a compreender alguns pontos que não tínhamos entendido tão bem.

% -------------------------------------------------------------------------------------------
\end{document}